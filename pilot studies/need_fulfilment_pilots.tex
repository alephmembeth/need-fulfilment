\documentclass[egregdoesnotlikesansseriftitles]{scrartcl}

\usepackage[T1]{fontenc}
\usepackage[utf8]{inputenc}

\usepackage{amsmath}
\usepackage{amssymb}
\usepackage{authblk}
   \renewcommand\Affilfont{\small}
\usepackage[style=german]{csquotes}
\usepackage{dcolumn}
   \newcolumntype{d}[1]{D{.}{.}{#1}}
\usepackage{float}
\usepackage{graphicx}
\usepackage[hidelinks]{hyperref}
   \urlstyle{rm}
\usepackage[authoryear]{natbib}
\usepackage{url}

\setcitestyle{aysep={}}

\deffootnote{1.5em}{1em}{\makebox[1.5em][l]{\thefootnotemark}}
   \setlength{\skip\footins}{1.5em}
   \setlength{\footnotesep}{1em}

\begin{document}

\clearpage
\appendix


%%%%%%%%%%%
% PILOT 1 %
%%%%%%%%%%%
\section{First Pilot Study}\label{sec:app_pilot_1}
\subsection*{Introduction}
Bitte stellen Sie sich Folgendes vor:

Familie Müller hat einen bestimmten Bedarf an Wohnraum.
Das heißt, sie braucht eine Wohnung mit einer gewissen Quadratmeterzahl, um annehmbar leben zu können.
Wenn sie genau diese Quadratmeter bekommt, ist ihr Bedarf an Wohnraum exakt erfüllt.
Wenn sie weniger bekommt, ist ihr Bedarf untererfüllt.
Wenn sie mehr bekommt, ist ihr Bedarf übererfüllt.
Familie Müller benötigt 100 Quadratmeter.
Auf der nächsten Seite werden wir Ihnen verschiedene Fälle zeigen, in denen die Wohnung, die Familie Müller tatsächlich bekommt, unterschiedlich groß ist.

Wir bitten Sie, selbst einzuschätzen, wie weit der Bedarf von Familie Müller in diesen Fällen erfüllt ist.
Dazu können Sie eine Zahl angeben.
0 bedeutet, dass der Bedarf exakt erfüllt ist.
Negative Werte bedeuten, dass der Bedarf untererfüllt ist.
Positive Werte bedeuten, dass der Bedarf übererfüllt ist.
Je stärker der Bedarf unter- oder übererfüllt ist, umso höher sollten die Werte sein.
Treffen Sie Ihre Beurteilungen bitte entsprechend Ihrer ganz persönlichen Einschätzung.
Es gibt keine richtigen oder falschen Beurteilungen.


\subsection*{Task}
\textit{Note: All cases were presented on one screen in ascending order. $i \in {1, 2, 3, \ldots, 21}$; $j \in {0, 10, 20, \ldots, 200}$.}

\vspace{1em}
\textbf{Fall \textit{i}}

\vspace{1em}
\begin{tabular}{cc}\hline
   Bedarf      & Tatsächliche Größe   \\\hline\hline
   100 m$^2$   & $j$ m$^2$            \\\hline
\end{tabular}

\vspace{1em}
Erfüllung des Bedarfs:


\subsection*{Control Questions}
\noindent\textbf{Question 1:} Was haben Sie in den vorangegangenen Fällen eingeschätzt?
\begin{itemize}
   \item[$\square$] Wie schön der Wohnraum eingerichtet ist
   \item[$\square$] Wieviele Wohnungen es in einer Stadt gibt
   \item[$\square$] Wie weit der Bedarf an Wohnraum erfüllt ist
\end{itemize}

\vspace{1em}
\noindent\textbf{Question 2:} Was war die höchste tatsächliche Größe in den vorangegangenen Fällen?
\begin{itemize}
   \item[$\square$] 50
   \item[$\square$] 100
   \item[$\square$] 150
   \item[$\square$] 200
   \item[$\square$] 250
   \item[$\square$] 300
\end{itemize}


\subsection*{Sociodemographics}
\textit{Note: All questions were presented on one screen.}

\vspace{1em}
\noindent\textbf{Age:} Wie alt sind Sie?

\vspace{1em}
\noindent\textbf{Gender:} Mit welchem Geschlecht identifizieren Sie sich?
\begin{itemize}
   \item[$\square$] weiblich
   \item[$\square$] nichtbinär
   \item[$\square$] männlich
\end{itemize}

\vspace{1em}
\noindent\textbf{Income:} Wieviel verdienen Sie zur Zeit netto, also nach Abzug von Steuern, ungefähr im Monat?

\vspace{1em}
\noindent\textbf{Education:} Was ist Ihr höchster Bildungsabschluss?
\begin{itemize}
   \item[$\square$] Kein Schulabschluss
   \item[$\square$] Hauptschulabschluss
   \item[$\square$] Mittlere Reife (Realschulabschluss)
   \item[$\square$] Allgemeine Hochschulreife oder Fachhochschulreife (Abitur oder Fachabitur)
   \item[$\square$] Abgeschlossene Berufsausbildung
   \item[$\square$] Techniker oder Meister
   \item[$\square$] Bachelor (oder vergleichbar)
   \item[$\square$] Master (oder vergleichbar)
   \item[$\square$] Promotion
\end{itemize}

\vspace{1em}
\noindent\textbf{State:} In welchem Bundesland befindet sich zur Zeit Ihr Lebensmittelpunkt?
\begin{itemize}
   \item[$\square$] Baden-Württemberg
   \item[$\square$] Bayern
   \item[$\square$] Berlin
   \item[$\square$] Brandenburg
   \item[$\square$] Bremen
   \item[$\square$] Hamburg
   \item[$\square$] Hessen
   \item[$\square$] Mecklenburg-Vorpommern
   \item[$\square$] Niedersachsen
   \item[$\square$] Nordrhein-Westfalen
   \item[$\square$] Rheinland-Pfalz
   \item[$\square$] Saarland
   \item[$\square$] Sachsen
   \item[$\square$] Sachsen-Anhalt
   \item[$\square$] Schleswig-Holstein
   \item[$\square$] Thüringen
\end{itemize}

\vspace{1em}
\noindent\textbf{Living Space:} Wie viele Quadratmeter umfasst die Wohnfläche, auf der Sie zur Zeit leben, ungefähr?

\vspace{1em}
\noindent\textbf{Household Members:} Mit wievielen Personen wohnen Sie zur Zeit zusammen? Wenn Sie alleine wohnen, geben Sie bitte 1 ein.

\vspace{1em}
\noindent\textbf{Political Orientation:} In der Politik spricht man von \enquote{links} und \enquote{rechts}. Wo auf einer Skala von 1 (links) bis 7 (rechts) würden Sie sich selbst einordnen?
\begin{itemize}
   \item[$\square$] 1 (links)
   \item[$\square$] 2
   \item[$\square$] 3
   \item[$\square$] 4
   \item[$\square$] 5
   \item[$\square$] 6
   \item[$\square$] 7 (rechts)
\end{itemize}


\subsection*{Closing}
Klicken Sie zum Abschließen der Studie bitte unten auf \enquote{Absenden}. Wenn Sie Fragen zu dieser Studie haben, wenden Sie sich bitte an mark.siebel@uol.de.


%%%%%%%%%%%
% PILOT 2 %
%%%%%%%%%%%
\clearpage
\section{Second Pilot Study}\label{sec:app_pilot_2}
\subsection*{Introduction}
Bitte stellen Sie sich Folgendes vor:

Familie Müller hat einen bestimmten Bedarf an Wohnraum.
Das heißt, sie braucht eine Wohnung mit einer gewissen Quadratmeterzahl, um annehmbar leben zu können.
Wenn sie genau diese Quadratmeter bekommt, ist ihr Bedarf an Wohnraum exakt erfüllt.
Wenn sie weniger bekommt, ist ihr Bedarf untererfüllt.
Wenn sie mehr bekommt, ist ihr Bedarf übererfüllt.
Familie Müller benötigt 100 Quadratmeter.
Auf der nächsten Seite werden wir Ihnen verschiedene Fälle zeigen, in denen die Wohnung, die Familie Müller tatsächlich bekommt, unterschiedlich groß ist.

Wir bitten Sie, selbst einzuschätzen, wie weit der Bedarf von Familie Müller in diesen Fällen erfüllt ist.
Dazu können Sie eine Zahl angeben.
0 bedeutet, dass der Bedarf exakt erfüllt ist.
Negative Werte bedeuten, dass der Bedarf untererfüllt ist.
Positive Werte bedeuten, dass der Bedarf übererfüllt ist.
Je stärker der Bedarf unter- oder übererfüllt ist, umso höher sollten die Werte sein.
Treffen Sie Ihre Beurteilungen bitte entsprechend Ihrer ganz persönlichen Einschätzung.
Es gibt keine richtigen oder falschen Beurteilungen.


\subsection*{Task}
\textit{Note: All cases were presented on separate screens in randomized order. $i \in {0, 10, 20, \ldots, 200}$.}

\vspace{1em}
\begin{tabular}{cc}\hline
   Bedarf      & Tatsächliche Größe   \\\hline\hline
   100 m$^2$   & $i$ m$^2$            \\\hline
\end{tabular}

\vspace{1em}
Erfüllung des Bedarfs:


\subsection*{Control Questions}
\noindent\textbf{Question 1:} Was haben Sie in den vorangegangenen Fällen eingeschätzt?
\begin{itemize}
   \item[$\square$] Wie schön der Wohnraum eingerichtet ist
   \item[$\square$] Wieviele Wohnungen es in einer Stadt gibt
   \item[$\square$] Wie weit der Bedarf an Wohnraum erfüllt ist
\end{itemize}

\vspace{1em}
\noindent\textbf{Question 2:} Was war die höchste tatsächliche Größe in den vorangegangenen Fällen?
\begin{itemize}
   \item[$\square$] 50
   \item[$\square$] 100
   \item[$\square$] 150
   \item[$\square$] 200
   \item[$\square$] 250
   \item[$\square$] 300
\end{itemize}


\subsection*{Sociodemographics}
\textit{Note: All questions were presented on separate screens.}

\vspace{1em}
\noindent\textbf{Age:} Wie alt sind Sie?

\vspace{1em}
\noindent\textbf{Gender:} Mit welchem Geschlecht identifizieren Sie sich?
\begin{itemize}
   \item[$\square$] weiblich
   \item[$\square$] nichtbinär
   \item[$\square$] männlich
\end{itemize}

\vspace{1em}
\noindent\textbf{Income:} Wieviel verdienen Sie zur Zeit netto, also nach Abzug von Steuern, ungefähr im Monat?

\vspace{1em}
\noindent\textbf{Education:} Was ist Ihr höchster Bildungsabschluss?
\begin{itemize}
   \item[$\square$] Kein Schulabschluss
   \item[$\square$] Hauptschulabschluss
   \item[$\square$] Mittlere Reife (Realschulabschluss)
   \item[$\square$] Allgemeine Hochschulreife oder Fachhochschulreife (Abitur oder Fachabitur)
   \item[$\square$] Abgeschlossene Berufsausbildung
   \item[$\square$] Techniker oder Meister
   \item[$\square$] Bachelor (oder vergleichbar)
   \item[$\square$] Master (oder vergleichbar)
   \item[$\square$] Promotion
\end{itemize}

\vspace{1em}
\noindent\textbf{State:} In welchem Bundesland befindet sich zur Zeit Ihr Lebensmittelpunkt?
\begin{itemize}
   \item[$\square$] Baden-Württemberg
   \item[$\square$] Bayern
   \item[$\square$] Berlin
   \item[$\square$] Brandenburg
   \item[$\square$] Bremen
   \item[$\square$] Hamburg
   \item[$\square$] Hessen
   \item[$\square$] Mecklenburg-Vorpommern
   \item[$\square$] Niedersachsen
   \item[$\square$] Nordrhein-Westfalen
   \item[$\square$] Rheinland-Pfalz
   \item[$\square$] Saarland
   \item[$\square$] Sachsen
   \item[$\square$] Sachsen-Anhalt
   \item[$\square$] Schleswig-Holstein
   \item[$\square$] Thüringen
\end{itemize}

\vspace{1em}
\noindent\textbf{Living Space:} Wie viele Quadratmeter umfasst die Wohnfläche, auf der Sie zur Zeit leben, ungefähr?

\vspace{1em}
\noindent\textbf{Household Members:} Mit wievielen Personen wohnen Sie zur Zeit zusammen? Wenn Sie alleine wohnen, geben Sie bitte 1 ein.

\vspace{1em}
\noindent\textbf{Political Orientation:} In der Politik spricht man von \enquote{links} und \enquote{rechts}. Wo auf einer Skala von 1 (links) bis 7 (rechts) würden Sie sich selbst einordnen?
\begin{itemize}
   \item[$\square$] 1 (links)
   \item[$\square$] 2
   \item[$\square$] 3
   \item[$\square$] 4
   \item[$\square$] 5
   \item[$\square$] 6
   \item[$\square$] 7 (rechts)
\end{itemize}


\subsection*{Closing}
Klicken Sie zum Abschließen der Studie bitte unten auf \enquote{Absenden}. Wenn Sie Fragen zu dieser Studie haben, wenden Sie sich bitte an mark.siebel@uol.de.


%%%%%%%%%%%
% PILOT 3 %
%%%%%%%%%%%
\section{Third Pilot Study}\label{sec:app_pilot_3}
\subsection*{Introduction}
Bitte stellen Sie sich Folgendes vor:

Familie Müller hat einen bestimmten Bedarf an Wohnraum.
Das heißt, sie braucht eine Wohnung mit einer gewissen Quadratmeterzahl, um annehmbar leben zu können.
Wenn sie genau diese Quadratmeter bekommt, ist ihr Bedarf an Wohnraum exakt erfüllt.
Wenn sie weniger bekommt, ist ihr Bedarf untererfüllt.
Wenn sie mehr bekommt, ist ihr Bedarf übererfüllt.
Familie Müller benötigt 100 Quadratmeter.
Auf der nächsten Seite werden wir Ihnen verschiedene Fälle zeigen, in denen die Wohnung, die Familie Müller tatsächlich bekommt, unterschiedlich groß ist.

Wir bitten Sie, selbst einzuschätzen, wie gerecht die Quadratmeterzahl ist, die Familie Müller in diesen Fällen bekommt.
Dazu können Sie eine Zahl angeben.
0 bedeutet, dass die Zuteilung gerecht ist.
Negative Werte bedeuten, dass die Zuteilung weniger als gerecht ist.
Positive Werte bedeuten, dass die Zuteilung mehr als gerecht ist.
Je stärker die Zuteilung weniger oder mehr als gerecht ist, umso höher sollten die Werte sein.
Treffen Sie Ihre Beurteilungen bitte entsprechend Ihrer ganz persönlichen Einschätzung.
Es gibt keine richtigen oder falschen Beurteilungen.


\subsection*{Task}
\textit{Note: All cases were presented on one screen in ascending order. $i \in {0, 10, 20, \ldots, 200}$.}

\vspace{1em}
\begin{tabular}{cc}\hline
   Bedarf      & Tatsächliche Größe   \\\hline\hline
   100 m$^2$   & $i$ m$^2$            \\\hline
\end{tabular}

\vspace{1em}
Gerechtigkeit der Zuteilung:


\subsection*{Control Questions}
\noindent\textbf{Question 1:} Was haben Sie in den vorangegangenen Fällen eingeschätzt?
\begin{itemize}
   \item[$\square$] Wie schön der Wohnraum eingerichtet ist
   \item[$\square$] Wieviele Wohnungen es in einer Stadt gibt
   \item[$\square$] Wie weit der Bedarf an Wohnraum erfüllt ist
\end{itemize}

\vspace{1em}
\noindent\textbf{Question 2:} Was war die höchste tatsächliche Größe in den vorangegangenen Fällen?
\begin{itemize}
   \item[$\square$] 50
   \item[$\square$] 100
   \item[$\square$] 150
   \item[$\square$] 200
   \item[$\square$] 250
   \item[$\square$] 300
\end{itemize}


\subsection*{Sociodemographics}
\textit{Note: All questions were presented on one screen.}

\vspace{1em}
\noindent\textbf{Age:} Wie alt sind Sie?

\vspace{1em}
\noindent\textbf{Gender:} Mit welchem Geschlecht identifizieren Sie sich?
\begin{itemize}
   \item[$\square$] weiblich
   \item[$\square$] nichtbinär
   \item[$\square$] männlich
\end{itemize}

\vspace{1em}
\noindent\textbf{Income:} Wieviel verdienen Sie zur Zeit netto, also nach Abzug von Steuern, ungefähr im Monat?

\vspace{1em}
\noindent\textbf{Education:} Was ist Ihr höchster Bildungsabschluss?
\begin{itemize}
   \item[$\square$] Kein Schulabschluss
   \item[$\square$] Hauptschulabschluss
   \item[$\square$] Mittlere Reife (Realschulabschluss)
   \item[$\square$] Allgemeine Hochschulreife oder Fachhochschulreife (Abitur oder Fachabitur)
   \item[$\square$] Abgeschlossene Berufsausbildung
   \item[$\square$] Techniker oder Meister
   \item[$\square$] Bachelor (oder vergleichbar)
   \item[$\square$] Master (oder vergleichbar)
   \item[$\square$] Promotion
\end{itemize}

\vspace{1em}
\noindent\textbf{State:} In welchem Bundesland befindet sich zur Zeit Ihr Lebensmittelpunkt?
\begin{itemize}
   \item[$\square$] Baden-Württemberg
   \item[$\square$] Bayern
   \item[$\square$] Berlin
   \item[$\square$] Brandenburg
   \item[$\square$] Bremen
   \item[$\square$] Hamburg
   \item[$\square$] Hessen
   \item[$\square$] Mecklenburg-Vorpommern
   \item[$\square$] Niedersachsen
   \item[$\square$] Nordrhein-Westfalen
   \item[$\square$] Rheinland-Pfalz
   \item[$\square$] Saarland
   \item[$\square$] Sachsen
   \item[$\square$] Sachsen-Anhalt
   \item[$\square$] Schleswig-Holstein
   \item[$\square$] Thüringen
\end{itemize}

\vspace{1em}
\noindent\textbf{Living Space:} Wie viele Quadratmeter umfasst die Wohnfläche, auf der Sie zur Zeit leben, ungefähr?

\vspace{1em}
\noindent\textbf{Household Members:} Mit wievielen Personen wohnen Sie zur Zeit zusammen? Wenn Sie alleine wohnen, geben Sie bitte 1 ein.

\vspace{1em}
\noindent\textbf{Political Orientation:} In der Politik spricht man von \enquote{links} und \enquote{rechts}. Wo auf einer Skala von 1 (links) bis 7 (rechts) würden Sie sich selbst einordnen?
\begin{itemize}
   \item[$\square$] 1 (links)
   \item[$\square$] 2
   \item[$\square$] 3
   \item[$\square$] 4
   \item[$\square$] 5
   \item[$\square$] 6
   \item[$\square$] 7 (rechts)
\end{itemize}


\subsection*{Closing}
Klicken Sie zum Abschließen der Studie bitte unten auf \enquote{Absenden}. Wenn Sie Fragen zu dieser Studie haben, wenden Sie sich bitte an mark.siebel@uol.de.


%%%%%%%%%%%
% PILOT 4 %
%%%%%%%%%%%
\section{Fourth Pilot Study}\label{sec:app_pilot_4}
\subsection*{Introduction}
Bitte stellen Sie sich Folgendes vor:

Familie Müller und Familie Schneider haben jeweils einen bestimmten Bedarf an Wohnraum.
Das heißt, sie brauchen eine Wohnung mit einer gewissen Quadratmeterzahl, um annehmbar leben zu können.

Auf den nächsten Seiten werden wir Ihnen verschiedene Fälle zeigen.
Von Fall zu Fall unterscheidet sich, wie viel Wohnraum Familie Müller und Familie Schneider benötigen und wie viel sie tatsächlich haben.

Wir bitten Sie, für jeden Fall anzugeben, ob beide Familien gleich ausreichend [gerecht] versorgt sind oder ob eine der Familien besser [gerechter] versorgt ist als die andere.
Außerdem bitten wir Sie, einzuschätzen, wie ausreichend [gerecht] beide Familien im jeweiligen Fall versorgt sind.
Dazu sollen Sie für beide jeweils eine Zahl angeben.
$0$ bedeutet, dass die Zuteilung ausreichend [gerecht] ist.
Negative Werte bedeuten, dass die Zuteilung weniger als ausreichend [gerecht] ist.
Positive Werte bedeuten, dass die Zuteilung mehr als ausreichend [gerecht] ist.
Je stärker die Zuteilung weniger oder mehr als ausreichend [gerecht] ist, umso weiter sollten die Werte von $0$ entfernt sein.
Treffen Sie Ihre Beurteilung bitte entsprechend Ihrer ganz persönlichen Einschätzung.
Es gibt keine richtigen oder falschen Beurteilungen.


\subsection*{Tasks}
\textit{Note: All cases were presented on separate screens in randomised order. Below each table, the situation was verbally sumarized (\enquote{Familie Müller benötigt v\,m$^2$ und hat w\,m$^2$. Familie Schneider benötigt x\,m$^2$ und hat y\,m$^2$.}). At the bottom of the screen, Questions 1 and 2 were shown.}

\vspace{1em}
\begin{tabular}{lcc}\hline
                      & Familie Müller   & Familie Schneider   \\\hline\hline
   \multicolumn{3}{c}{\textit{Case 1}}\\\hline
   Benötigt (m$^2$)   &  40              & 200                 \\
   Hat (m$^2$)        &  20              & 180                 \\\hline
   \multicolumn{3}{c}{\textit{Case 2}}\\\hline
   Benötigt (m$^2$)   &  60              & 200                 \\
   Hat (m$^2$)        &  20              & 160                 \\\hline
   \multicolumn{3}{c}{\textit{Case 3}}\\\hline
   Benötigt (m$^2$)   &  80              & 200                 \\
   Hat (m$^2$)        &  20              & 140                 \\\hline
   \multicolumn{3}{c}{\textit{Case 4}}\\\hline
   Benötigt (m$^2$)   &  25              & 175                 \\
   Hat (m$^2$)        &  20              & 140                 \\\hline
   \multicolumn{3}{c}{\textit{Case 5}}\\\hline
   Benötigt (m$^2$)   &  40              & 200                 \\
   Hat (m$^2$)        &  24              & 120                 \\\hline
   \multicolumn{3}{c}{\textit{Case 6}}\\\hline
   Benötigt (m$^2$)   &  40              & 200                 \\
   Hat (m$^2$)        &  16              &  80                 \\\hline
\end{tabular}

\vspace{1em}
\noindent\textbf{Question 1:} Welche der folgenden Aussagen trifft zu?
\begin{itemize}
   \item[$\square$] Familie Müller und Familie Schneider sind gleich ausreichend [gerecht] versorgt.
   \item[$\square$] Familie Müller ist besser [gerechter] versorgt als Familie Schneider.
   \item[$\square$] Familie Schneider ist besser [gerechter] versorgt als Familie Müller.
\end{itemize}

\vspace{1em}
\noindent\textbf{Question 2:} Bitte schätzen Sie ein, wie ausreichend [gerecht] Familie Müller und Familie Schneider in diesem Fall versorgt sind.

Geben Sie dazu jeweils für beide eine Zahl an.
$0$ bedeutet, dass die Zuteilung ausreichend [gerecht] ist.
Negative Werte bedeuten, dass die Zuteilung weniger als ausreichend [gerecht] ist.
Positive Werte bedeuten, dass die Zuteilung mehr als ausreichend [gerecht] ist.
Je stärker die Zuteilung weniger oder mehr als ausreichend [gerecht] ist, umso weiter sollten die Werte von $0$ entfernt sein.
Treffen Sie Ihre Beurteilung bitte entsprechend Ihrer ganz persönlichen Einschätzung.
Es gibt keine richtigen oder falschen Beurteilungen.

\begin{itemize}
   \item[--] Familie Müller:\ \rule{2cm}{0.4pt}
   \item[--] Familie Schneider:\ \rule{2cm}{0.4pt}
\end{itemize}


\subsection*{Control Questions}
\noindent\textbf{Question 1:} Was haben Sie in den vorangegangenen Fällen eingeschätzt?
\begin{itemize}
   \item[$\square$] Wie schön der Wohnraum eingerichtet ist
   \item[$\square$] Wieviele Wohnungen es in einer Stadt gibt
   \item[$\square$] Wie ausreichend [gerecht] die Zuteilung an Wohnraum ist
\end{itemize}

\vspace{1em}
\noindent\textbf{Question 2:} Was war die höchste tatsächliche Größe in den vorangegangenen Fällen?
\begin{itemize}
   \item[$\square$] 50
   \item[$\square$] 100
   \item[$\square$] 150
   \item[$\square$] 200
   \item[$\square$] 250
   \item[$\square$] 300
\end{itemize}


\subsection*{Sociodemographics}
\textit{Note: All questions were presented on one screen.}

\vspace{1em}
\noindent\textbf{Age:} Wie alt sind Sie?

\vspace{1em}
\noindent\textbf{Gender:} Mit welchem Geschlecht identifizieren Sie sich?
\begin{itemize}
   \item[$\square$] weiblich
   \item[$\square$] nichtbinär
   \item[$\square$] männlich
\end{itemize}

\vspace{1em}
\noindent\textbf{Income:} Wieviel verdienen Sie zur Zeit netto, also nach Abzug von Steuern, ungefähr im Monat?

\vspace{1em}
\noindent\textbf{Education:} Was ist Ihr höchster Bildungsabschluss?
\begin{itemize}
   \item[$\square$] Kein Schulabschluss
   \item[$\square$] Hauptschulabschluss
   \item[$\square$] Mittlere Reife (Realschulabschluss)
   \item[$\square$] Allgemeine Hochschulreife oder Fachhochschulreife (Abitur oder Fachabitur)
   \item[$\square$] Abgeschlossene Berufsausbildung
   \item[$\square$] Techniker oder Meister
   \item[$\square$] Bachelor (oder vergleichbar)
   \item[$\square$] Master (oder vergleichbar)
   \item[$\square$] Promotion
\end{itemize}

\vspace{1em}
\noindent\textbf{State:} In welchem Bundesland befindet sich zur Zeit Ihr Lebensmittelpunkt?
\begin{itemize}
   \item[$\square$] Baden-Württemberg
   \item[$\square$] Bayern
   \item[$\square$] Berlin
   \item[$\square$] Brandenburg
   \item[$\square$] Bremen
   \item[$\square$] Hamburg
   \item[$\square$] Hessen
   \item[$\square$] Mecklenburg-Vorpommern
   \item[$\square$] Niedersachsen
   \item[$\square$] Nordrhein-Westfalen
   \item[$\square$] Rheinland-Pfalz
   \item[$\square$] Saarland
   \item[$\square$] Sachsen
   \item[$\square$] Sachsen-Anhalt
   \item[$\square$] Schleswig-Holstein
   \item[$\square$] Thüringen
\end{itemize}

\vspace{1em}
\noindent\textbf{Living Space:} Wie viele Quadratmeter umfasst die Wohnfläche, auf der Sie zur Zeit leben, ungefähr?

\vspace{1em}
\noindent\textbf{Household Members:} Mit wievielen Personen wohnen Sie zur Zeit zusammen? Wenn Sie alleine wohnen, geben Sie bitte 1 ein.

\vspace{1em}
\noindent\textbf{Political Orientation:} In der Politik spricht man von \enquote{links} und \enquote{rechts}. Wo auf einer Skala von 1 (links) bis 7 (rechts) würden Sie sich selbst einordnen?
\begin{itemize}
   \item[$\square$] 1 (links)
   \item[$\square$] 2
   \item[$\square$] 3
   \item[$\square$] 4
   \item[$\square$] 5
   \item[$\square$] 6
   \item[$\square$] 7 (rechts)
\end{itemize}


\subsection*{Closing}
\noindent\textbf{Page 1:} Klicken Sie zum Abschließen der Studie bitte unten auf \enquote{Absenden}. Auf der nächsten Seite finden Sie Informationen zur Teilnahme an der Gutscheinverlosung..

\vspace{1em}
\noindent\textbf{Page 2:} Vielen Dank für Ihre Teilnahme.

Wenn Sie an der Gutscheinverlosung teilnehmen möchten, klicken Sie bitte auf diesen Link und geben Sie dort Ihre E-Mail-Adresse an.

Schreiben Sie uns gerne eine E-Mail, falls Sie Fragen haben oder über die Ergebnisse der Umfrage informiert werden möchten, sobald sie veröffentlicht sind.

\vspace{0.5em}
\noindent Prof. Dr. Mark Siebel\\
(mark.siebel@uol.de)

\vspace{0.5em}
\noindent Dr. Alexander Max Bauer\\
(alexander.max.bauer@uol.de)


\subsection*{Raffle}
Wenn Sie an der Verlosung teilnehmen möchten, geben Sie bitte hier Ihre E-Mail-Adresse an und klicken Sie abschließend auf \enquote{Absenden}.


\end{document}
