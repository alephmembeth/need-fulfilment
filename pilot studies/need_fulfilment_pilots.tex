\documentclass[egregdoesnotlikesansseriftitles]{scrartcl}

\usepackage[T1]{fontenc}
\usepackage[utf8]{inputenc}

\usepackage{amsmath}
\usepackage{amssymb}
\usepackage{authblk}
   \renewcommand\Affilfont{\small}
\usepackage[style=german]{csquotes}
\usepackage{dcolumn}
   \newcolumntype{d}[1]{D{.}{.}{#1}}
\usepackage{float}
\usepackage{graphicx}
\usepackage[hidelinks]{hyperref}
   \urlstyle{rm}
\usepackage[authoryear]{natbib}
\usepackage{url}

\setcitestyle{aysep={}}

\deffootnote{1.5em}{1em}{\makebox[1.5em][l]{\thefootnotemark}}
   \setlength{\skip\footins}{1.5em}
   \setlength{\footnotesep}{1em}

\begin{document}

\clearpage
\appendix


%%%%%%%%%%%
% PILOT 1 %
%%%%%%%%%%%
\section{First Pilot Study}\label{sec:app_pilot_1}
\subsection*{Introduction}
Bitte stellen Sie sich Folgendes vor:

Familie Müller hat einen bestimmten Bedarf an Wohnraum.
Das heißt, sie braucht eine Wohnung mit einer gewissen Quadratmeterzahl, um annehmbar leben zu können.
Wenn sie genau diese Quadratmeter bekommt, ist ihr Bedarf an Wohnraum exakt erfüllt.
Wenn sie weniger bekommt, ist ihr Bedarf untererfüllt.
Wenn sie mehr bekommt, ist ihr Bedarf übererfüllt.
Familie Müller benötigt 100 Quadratmeter.
Auf der nächsten Seite werden wir Ihnen verschiedene Fälle zeigen, in denen die Wohnung, die Familie Müller tatsächlich bekommt, unterschiedlich groß ist.

Wir bitten Sie, selbst einzuschätzen, wie weit der Bedarf von Familie Müller in diesen Fällen erfüllt ist.
Dazu können Sie eine Zahl angeben.
0 bedeutet, dass der Bedarf exakt erfüllt ist.
Negative Werte bedeuten, dass der Bedarf untererfüllt ist.
Positive Werte bedeuten, dass der Bedarf übererfüllt ist.
Je stärker der Bedarf unter- oder übererfüllt ist, umso höher sollten die Werte sein.
Treffen Sie Ihre Beurteilungen bitte entsprechend Ihrer ganz persönlichen Einschätzung.
Es gibt keine richtigen oder falschen Beurteilungen.


\subsection*{Task}
\textit{Note: All cases were presented on one screen in ascending order. $i \in {1, 2, 3, \ldots, 21}$; $j \in {0, 10, 20, \ldots, 200}$.}

\vspace{1em}
\textbf{Fall \textit{i}}

\vspace{1em}
\begin{tabular}{cc}\hline
   Bedarf      & Tatsächliche Größe   \\\hline\hline
   100 m$^2$   & $j$ m$^2$            \\\hline
\end{tabular}

\vspace{1em}
Erfüllung des Bedarfs:


\subsection*{Control Questions}
\noindent\textbf{Question 1:} Was haben Sie in den vorangegangenen Fällen eingeschätzt?
\begin{itemize}
   \item[$\square$] Wie schön der Wohnraum eingerichtet ist
   \item[$\square$] Wieviele Wohnungen es in einer Stadt gibt
   \item[$\square$] Wie weit der Bedarf an Wohnraum erfüllt ist
\end{itemize}

\vspace{1em}
\noindent\textbf{Question 2:} Was war die höchste tatsächliche Größe in den vorangegangenen Fällen?
\begin{itemize}
   \item[$\square$] 50
   \item[$\square$] 100
   \item[$\square$] 150
   \item[$\square$] 200
   \item[$\square$] 250
   \item[$\square$] 300
\end{itemize}


\subsection*{Sociodemographics}
\textit{Note: All questions were presented on one screen.}

\vspace{1em}
\noindent\textbf{Age:} Wie alt sind Sie?

\vspace{1em}
\noindent\textbf{Gender:} Mit welchem Geschlecht identifizieren Sie sich?
\begin{itemize}
   \item[$\square$] weiblich
   \item[$\square$] nichtbinär
   \item[$\square$] männlich
\end{itemize}

\vspace{1em}
\noindent\textbf{Income:} Wieviel verdienen Sie zur Zeit netto, also nach Abzug von Steuern, ungefähr im Monat?

\vspace{1em}
\noindent\textbf{Education:} Was ist Ihr höchster Bildungsabschluss?
\begin{itemize}
   \item[$\square$] Kein Schulabschluss
   \item[$\square$] Hauptschulabschluss
   \item[$\square$] Mittlere Reife (Realschulabschluss)
   \item[$\square$] Allgemeine Hochschulreife oder Fachhochschulreife (Abitur oder Fachabitur)
   \item[$\square$] Abgeschlossene Berufsausbildung
   \item[$\square$] Techniker oder Meister
   \item[$\square$] Bachelor (oder vergleichbar)
   \item[$\square$] Master (oder vergleichbar)
   \item[$\square$] Promotion
\end{itemize}

\vspace{1em}
\noindent\textbf{State:} In welchem Bundesland befindet sich zur Zeit Ihr Lebensmittelpunkt?
\begin{itemize}
   \item[$\square$] Baden-Württemberg
   \item[$\square$] Bayern
   \item[$\square$] Berlin
   \item[$\square$] Brandenburg
   \item[$\square$] Bremen
   \item[$\square$] Hamburg
   \item[$\square$] Hessen
   \item[$\square$] Mecklenburg-Vorpommern
   \item[$\square$] Niedersachsen
   \item[$\square$] Nordrhein-Westfalen
   \item[$\square$] Rheinland-Pfalz
   \item[$\square$] Saarland
   \item[$\square$] Sachsen
   \item[$\square$] Sachsen-Anhalt
   \item[$\square$] Schleswig-Holstein
   \item[$\square$] Thüringen
\end{itemize}

\vspace{1em}
\noindent\textbf{Living Space:} Wie viele Quadratmeter umfasst die Wohnfläche, auf der Sie zur Zeit leben, ungefähr?

\vspace{1em}
\noindent\textbf{Household members:} Mit wievielen Personen wohnen Sie zur Zeit zusammen? Wenn Sie alleine wohnen, geben Sie bitte 1 ein.

\vspace{1em}
\noindent\textbf{Political Orientation:} In der Politik spricht man von \enquote{links} und \enquote{rechts}. Wo auf einer Skala von 1 (links) bis 7 (rechts) würden Sie sich selbst einordnen?
\begin{itemize}
   \item[$\square$] 1 (links)
   \item[$\square$] 2
   \item[$\square$] 3
   \item[$\square$] 4
   \item[$\square$] 5
   \item[$\square$] 6
   \item[$\square$] 7 (rechts)
\end{itemize}


\subsection*{Closing}
Klicken Sie zum Abschließen der Studie bitte unten auf \enquote{Absenden}. Wenn Sie Fragen zu dieser Studie haben, wenden Sie sich bitte an mark.siebel@uol.de.


%%%%%%%%%%%
% PILOT 2 %
%%%%%%%%%%%
\clearpage
\section{Second Pilot Study}\label{sec:app_pilot_2}
\subsection*{Introduction}
Bitte stellen Sie sich Folgendes vor:

Familie Müller hat einen bestimmten Bedarf an Wohnraum.
Das heißt, sie braucht eine Wohnung mit einer gewissen Quadratmeterzahl, um annehmbar leben zu können.
Wenn sie genau diese Quadratmeter bekommt, ist ihr Bedarf an Wohnraum exakt erfüllt.
Wenn sie weniger bekommt, ist ihr Bedarf untererfüllt.
Wenn sie mehr bekommt, ist ihr Bedarf übererfüllt.
Familie Müller benötigt 100 Quadratmeter.
Auf der nächsten Seite werden wir Ihnen verschiedene Fälle zeigen, in denen die Wohnung, die Familie Müller tatsächlich bekommt, unterschiedlich groß ist.

Wir bitten Sie, selbst einzuschätzen, wie weit der Bedarf von Familie Müller in diesen Fällen erfüllt ist.
Dazu können Sie eine Zahl angeben.
0 bedeutet, dass der Bedarf exakt erfüllt ist.
Negative Werte bedeuten, dass der Bedarf untererfüllt ist.
Positive Werte bedeuten, dass der Bedarf übererfüllt ist.
Je stärker der Bedarf unter- oder übererfüllt ist, umso höher sollten die Werte sein.
Treffen Sie Ihre Beurteilungen bitte entsprechend Ihrer ganz persönlichen Einschätzung.
Es gibt keine richtigen oder falschen Beurteilungen.


\subsection*{Task}
\textit{Note: All cases were presented on separate screens in randomized order. $i \in {0, 10, 20, \ldots, 200}$.}

\vspace{1em}
\begin{tabular}{cc}\hline
   Bedarf      & Tatsächliche Größe   \\\hline\hline
   100 m$^2$   & $i$ m$^2$            \\\hline
\end{tabular}

\vspace{1em}
Erfüllung des Bedarfs:


\subsection*{Control Questions}
\noindent\textbf{Question 1:} Was haben Sie in den vorangegangenen Fällen eingeschätzt?
\begin{itemize}
   \item[$\square$] Wie schön der Wohnraum eingerichtet ist
   \item[$\square$] Wieviele Wohnungen es in einer Stadt gibt
   \item[$\square$] Wie weit der Bedarf an Wohnraum erfüllt ist
\end{itemize}

\vspace{1em}
\noindent\textbf{Question 2:} Was war die höchste tatsächliche Größe in den vorangegangenen Fällen?
\begin{itemize}
   \item[$\square$] 50
   \item[$\square$] 100
   \item[$\square$] 150
   \item[$\square$] 200
   \item[$\square$] 250
   \item[$\square$] 300
\end{itemize}


\subsection*{Sociodemographics}
\textit{Note: All questions were presented on separate screens.}

\vspace{1em}
\noindent\textbf{Age:} Wie alt sind Sie?

\vspace{1em}
\noindent\textbf{Gender:} Mit welchem Geschlecht identifizieren Sie sich?
\begin{itemize}
   \item[$\square$] weiblich
   \item[$\square$] nichtbinär
   \item[$\square$] männlich
\end{itemize}

\vspace{1em}
\noindent\textbf{Income:} Wieviel verdienen Sie zur Zeit netto, also nach Abzug von Steuern, ungefähr im Monat?

\vspace{1em}
\noindent\textbf{Education:} Was ist Ihr höchster Bildungsabschluss?
\begin{itemize}
   \item[$\square$] Kein Schulabschluss
   \item[$\square$] Hauptschulabschluss
   \item[$\square$] Mittlere Reife (Realschulabschluss)
   \item[$\square$] Allgemeine Hochschulreife oder Fachhochschulreife (Abitur oder Fachabitur)
   \item[$\square$] Abgeschlossene Berufsausbildung
   \item[$\square$] Techniker oder Meister
   \item[$\square$] Bachelor (oder vergleichbar)
   \item[$\square$] Master (oder vergleichbar)
   \item[$\square$] Promotion
\end{itemize}

\vspace{1em}
\noindent\textbf{State:} In welchem Bundesland befindet sich zur Zeit Ihr Lebensmittelpunkt?
\begin{itemize}
   \item[$\square$] Baden-Württemberg
   \item[$\square$] Bayern
   \item[$\square$] Berlin
   \item[$\square$] Brandenburg
   \item[$\square$] Bremen
   \item[$\square$] Hamburg
   \item[$\square$] Hessen
   \item[$\square$] Mecklenburg-Vorpommern
   \item[$\square$] Niedersachsen
   \item[$\square$] Nordrhein-Westfalen
   \item[$\square$] Rheinland-Pfalz
   \item[$\square$] Saarland
   \item[$\square$] Sachsen
   \item[$\square$] Sachsen-Anhalt
   \item[$\square$] Schleswig-Holstein
   \item[$\square$] Thüringen
\end{itemize}

\vspace{1em}
\noindent\textbf{Living Space:} Wie viele Quadratmeter umfasst die Wohnfläche, auf der Sie zur Zeit leben, ungefähr?

\vspace{1em}
\noindent\textbf{Household members:} Mit wievielen Personen wohnen Sie zur Zeit zusammen? Wenn Sie alleine wohnen, geben Sie bitte 1 ein.

\vspace{1em}
\noindent\textbf{Political Orientation:} In der Politik spricht man von \enquote{links} und \enquote{rechts}. Wo auf einer Skala von 1 (links) bis 7 (rechts) würden Sie sich selbst einordnen?
\begin{itemize}
   \item[$\square$] 1 (links)
   \item[$\square$] 2
   \item[$\square$] 3
   \item[$\square$] 4
   \item[$\square$] 5
   \item[$\square$] 6
   \item[$\square$] 7 (rechts)
\end{itemize}


\subsection*{Closing}
Klicken Sie zum Abschließen der Studie bitte unten auf \enquote{Absenden}. Wenn Sie Fragen zu dieser Studie haben, wenden Sie sich bitte an mark.siebel@uol.de.


%%%%%%%%%%%
% PILOT 3 %
%%%%%%%%%%%
\section{Third Pilot Study}\label{sec:app_pilot_3}
\subsection*{Introduction}
Bitte stellen Sie sich Folgendes vor:

Familie Müller hat einen bestimmten Bedarf an Wohnraum.
Das heißt, sie braucht eine Wohnung mit einer gewissen Quadratmeterzahl, um annehmbar leben zu können.
Wenn sie genau diese Quadratmeter bekommt, ist ihr Bedarf an Wohnraum exakt erfüllt.
Wenn sie weniger bekommt, ist ihr Bedarf untererfüllt.
Wenn sie mehr bekommt, ist ihr Bedarf übererfüllt.
Familie Müller benötigt 100 Quadratmeter.
Auf der nächsten Seite werden wir Ihnen verschiedene Fälle zeigen, in denen die Wohnung, die Familie Müller tatsächlich bekommt, unterschiedlich groß ist.

Wir bitten Sie, selbst einzuschätzen, wie gerecht die Quadratmeterzahl ist, die Familie Müller in diesen Fällen bekommt.
Dazu können Sie eine Zahl angeben.
0 bedeutet, dass die Zuteilung gerecht ist.
Negative Werte bedeuten, dass die Zuteilung weniger als gerecht ist.
Positive Werte bedeuten, dass die Zuteilung mehr als gerecht ist.
Je stärker die Zuteilung weniger oder mehr als gerecht ist, umso höher sollten die Werte sein.
Treffen Sie Ihre Beurteilungen bitte entsprechend Ihrer ganz persönlichen Einschätzung.
Es gibt keine richtigen oder falschen Beurteilungen.


\subsection*{Task}
\textit{Note: All cases were presented on one screen in ascending order. $i \in {0, 10, 20, \ldots, 200}$.}

\vspace{1em}
\begin{tabular}{cc}\hline
   Bedarf      & Tatsächliche Größe   \\\hline\hline
   100 m$^2$   & $i$ m$^2$            \\\hline
\end{tabular}

\vspace{1em}
Gerechtigkeit der Zuteilung:


\subsection*{Control Questions}
\noindent\textbf{Question 1:} Was haben Sie in den vorangegangenen Fällen eingeschätzt?
\begin{itemize}
   \item[$\square$] Wie schön der Wohnraum eingerichtet ist
   \item[$\square$] Wieviele Wohnungen es in einer Stadt gibt
   \item[$\square$] Wie weit der Bedarf an Wohnraum erfüllt ist
\end{itemize}

\vspace{1em}
\noindent\textbf{Question 2:} Was war die höchste tatsächliche Größe in den vorangegangenen Fällen?
\begin{itemize}
   \item[$\square$] 50
   \item[$\square$] 100
   \item[$\square$] 150
   \item[$\square$] 200
   \item[$\square$] 250
   \item[$\square$] 300
\end{itemize}


\subsection*{Sociodemographics}
\textit{Note: All questions were presented on one screen.}

\vspace{1em}
\noindent\textbf{Age:} Wie alt sind Sie?

\vspace{1em}
\noindent\textbf{Gender:} Mit welchem Geschlecht identifizieren Sie sich?
\begin{itemize}
   \item[$\square$] weiblich
   \item[$\square$] nichtbinär
   \item[$\square$] männlich
\end{itemize}

\vspace{1em}
\noindent\textbf{Income:} Wieviel verdienen Sie zur Zeit netto, also nach Abzug von Steuern, ungefähr im Monat?

\vspace{1em}
\noindent\textbf{Education:} Was ist Ihr höchster Bildungsabschluss?
\begin{itemize}
   \item[$\square$] Kein Schulabschluss
   \item[$\square$] Hauptschulabschluss
   \item[$\square$] Mittlere Reife (Realschulabschluss)
   \item[$\square$] Allgemeine Hochschulreife oder Fachhochschulreife (Abitur oder Fachabitur)
   \item[$\square$] Abgeschlossene Berufsausbildung
   \item[$\square$] Techniker oder Meister
   \item[$\square$] Bachelor (oder vergleichbar)
   \item[$\square$] Master (oder vergleichbar)
   \item[$\square$] Promotion
\end{itemize}

\vspace{1em}
\noindent\textbf{State:} In welchem Bundesland befindet sich zur Zeit Ihr Lebensmittelpunkt?
\begin{itemize}
   \item[$\square$] Baden-Württemberg
   \item[$\square$] Bayern
   \item[$\square$] Berlin
   \item[$\square$] Brandenburg
   \item[$\square$] Bremen
   \item[$\square$] Hamburg
   \item[$\square$] Hessen
   \item[$\square$] Mecklenburg-Vorpommern
   \item[$\square$] Niedersachsen
   \item[$\square$] Nordrhein-Westfalen
   \item[$\square$] Rheinland-Pfalz
   \item[$\square$] Saarland
   \item[$\square$] Sachsen
   \item[$\square$] Sachsen-Anhalt
   \item[$\square$] Schleswig-Holstein
   \item[$\square$] Thüringen
\end{itemize}

\vspace{1em}
\noindent\textbf{Living Space:} Wie viele Quadratmeter umfasst die Wohnfläche, auf der Sie zur Zeit leben, ungefähr?

\vspace{1em}
\noindent\textbf{Household members:} Mit wievielen Personen wohnen Sie zur Zeit zusammen? Wenn Sie alleine wohnen, geben Sie bitte 1 ein.

\vspace{1em}
\noindent\textbf{Political Orientation:} In der Politik spricht man von \enquote{links} und \enquote{rechts}. Wo auf einer Skala von 1 (links) bis 7 (rechts) würden Sie sich selbst einordnen?
\begin{itemize}
   \item[$\square$] 1 (links)
   \item[$\square$] 2
   \item[$\square$] 3
   \item[$\square$] 4
   \item[$\square$] 5
   \item[$\square$] 6
   \item[$\square$] 7 (rechts)
\end{itemize}


\subsection*{Closing}
Klicken Sie zum Abschließen der Studie bitte unten auf \enquote{Absenden}. Wenn Sie Fragen zu dieser Studie haben, wenden Sie sich bitte an mark.siebel@uol.de.


\end{document}
